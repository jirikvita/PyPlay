%%%%%%%%%%%%%%%%%%%%%%%%%%
%%%%%%%%%%%%%%%%%%%%%%%%%%

\documentclass[a4paper,twoside]{article}
\usepackage[utf8]{inputenc}

\author{Martin Fafejta, Jiří Kvita}
\title{Výsledky dotazníku o dění na UP}

%\usepackage[latin2]{inputenc} % pro iso8859-2
%\usepackage[cp1250]{inputenc} % pro win1250
\usepackage{graphicx,epsfig}
\usepackage{slashed}
\usepackage{color}
\usepackage[czech]{babel}
\textwidth 145mm
\textheight 24cm
\topmargin -0.5cm
\evensidemargin 0.725cm
\oddsidemargin 0.725cm 

%%%%%%%%%%%%%%%%%%%%%%%%%%
%%%%%%%%%%%%%%%%%%%%%%%%%%
%%%%%%%%%%%%%%%%%%%%%%%%%%

\include{defs}

\usepackage{url}
\usepackage{hyperref}
\usepackage{ dsfont }
\usepackage{siunitx}

% \usepackage[utf8]{inputenc}

%%%%%%%%%%%%%%%%%%%%%%%%%%
%%%%%%%%%%%%%%%%%%%%%%%%%%
%%%%%%%%%%%%%%%%%%%%%%%%%%

\begin{document}
\maketitle

\tableofcontents
\newpage

\section{Vznik ankety}

Na základě \htmladdnormallink{\textcolor{blue}{\underline{ohlášení záměru uskutečnit anketu}}}{https://www.zurnal.upol.cz/nc/zprava-int/clanek/otazky-do-pruzkumu/} ze dne 31.1.2020 provedl následujících týdnech hlavní výběr otázek J.~Kvita a dále úpravu jejich formulací, pořadí odpovědí a velmi často také rozdělení na více otázek M.~Fafejta. Vedení UP se aktivně ke vzniku ankety kromě kratšího komentáře~\footnote{P. Bilík, 4. únor 2020, v 11:08: ``Můžu konzultovat s Personálním oddělením, jinak nejsme schopni dotazník efektivně rozesílat. Obecně jsem pro, pokud se podaří zachovat neutrální znění i zprostředkování cílové skupině.''}, nepostavilo a nekontaktovalo nás.

Anketa byla ohlášena v sobotu 22.2.2020 \htmladdnormallink{\textcolor{blue}{\underline{odkazem}}}{https://forms.gle/gjpwghZehjD6pYhBA} v dopoledních hodinách \htmladdnormallink{\textcolor{blue}{\underline{na UP reflexi}}}{https://www.zurnal.upol.cz/nc/reflexe/zprava-int/clanek/dotaznik-o-deni-na-up/} a dále na \htmladdnormallink{\textcolor{blue}{\underline{dŽurnálu}}}{https://www.dzurnal.cz/index.php/2020/02/22/dotaznik-o-deni-na-up/}. Platformou ankety byla zvolena služba Google Forms, tj. stejná jako např. poslední anketa vedení UP ohledně \htmladdnormallink{\textcolor{blue}{\underline{společenské odpovědnosti UP}}}{https://docs.google.com/forms/d/e/1FAIpQLSfZWWqXgJ3FopOOAhVo8jLqp4MB4ak-IGnJqUEd1DQt8nggUA/viewform}. Anketa byla uzavřena 1.3.2020 ve 23:59.

Vyšší podíl respondentů mohl jednoduše zajistit pan prorektor Petr Bilík, který ale bohužel odmítl spolupracovat se šířením ankety UP informačními kanály, a dále bohužel na UP reflexi uvedl tak trochu návodný komentář, že anketu lze vícenásobně vyplnit (stejně ale jako společenskoodpovědní anketu UP). Toho v jejím závěru opravdu někdo z části zneužil, ale jen s malým dopadem na výsledky, viz níže.

Náš dotazník si nekladl za cíl být reprezentativním výzkumem, je to anketa, z které z principu nemůžeme získat zobecnitelná data, nicméně odráží názory přes 700 respondentů, které jsme oslovili jak na UP na platformě UP reflexe tak i mimo UP. Jak bylo zmíněno, vedení UP nám bohužel odmítlo pomoci se šířením oficiálními univerzitními kanály, které by mohly zajistit ještě vyšší míru zastoupení dalších fakult, věrohodných odpovědí a širšího spektra respondentů.

Odpovědi jsme však podle výsledků a komentářů získali evidentně z celého názorového spektra. Někteří tazatelé měli samozřejmě jiný názor na výběr a zmínili i svůj názor o zaujatosti některých položených otázek. Netvrdíme tedy, že naše anketa a práce byla kompletní či bez chyby, nicméně záměr anketu vytvořit a nabídka možnosti aktivně se na její tvorbě podílet byl jasně a dlouho avizován na UP reflexi. Anketu spolu vytvořili J. Kvita (PřF) a M. Fafejta (FF), kteří se nebrání další spolupráci na tvorbě ankety podporované RUP, pokud s iniciativou začne samo vedení.


\section{Věrohodnost odpovědí}
\label{sec:Sel}
Pro AS UP jsme ve středu ráno 26.2.2020 zaslali analýzu \htmladdnormallink{\textcolor{blue}{\underline{průběžných výsledků ankety}}}{http://jointlab.upol.cz/kvita/dotaznik/results_prelim_for_ASUP/}, včetně porovnání výsledků s a bez komentáře.
Motivací pro tuto studii byla skutečnost, že pole komentář nebylo do cca 500 sesbíraných odpovědí povinné, a respondenti je připojovali v cca 1/4 případů. Předpoklad byl takový, že po zavedení povinného pole komentáře na konci dotazníku by měla klesnout případná angažovanost možných opakovaně vyplňujích anonymních účastníků, jakkoli počáteční frekvence odpovědí spíše indikuje počáteční nadšení zájemců ankety se zúčastnit ihned po zveřejnění. 

Objevily se však jasné známky toho, že tzv. "trollení", tj. cílené vyplňování dotazníku s cílem ovlivnit výsledky, se objevilo od cca odpovědí 630 a dále, od dopoledne 28.2.2020, a to na základě krátkodobého nárůstu četnosti odpovědí v anketě, která již byla jinak nasycena, viz Obr.~\ref{fig:troll_indication}. Odstranili jsme odpovědi s identickými komentáři, často přicházející v pravidelných intervalech 1,5 minuty či jindy 5 min, a s ``konverzačním'' charakterem komentářů, což dále nazýváme jako ``odstranění trollingu''. Nadále sice zůstává podezření, že výsledky ke konci ankety jsou z části nereprezentativní, ale ponecháváme je. Srovnání výsledků po odstranění trollingu (odstraněné 87 odpovědí) s případem, kdy byly jako základní úprava věrohodnosti jen odstraněny duplicitní komentáře (odstraněno 34 odpovědí) viz Doplněk~\ref{sec:trolling}.

Z důvodu trollingu již původní srovnání výsledků pro AS UP "s a bez komentáře" (viz Doplněk~\ref{sec:comments}) nesplňuje původní myšlenku testu zaujatosti. Respektujeme nicméně možnost, že se příznivci ústavu ankety začali účastnit až později, a výsledky od 630 do 757 ponecháváme v souboru, avšak po odstranění trollingu. 

% _____________________________________________________________________ %
\begin{figure}[!h]
\begin{tabular}{cc}
{  \includegraphics[width=0.49\textwidth]{ResultsNonTrolled/pdf_All/00.pdf} } & 
{  \includegraphics[width=0.49\textwidth]{ResultsFullTrolled/pdf_All/00.pdf}} \\
 non-Trolled & Trolled \\

\end{tabular}
\caption{Indikace trollingu v čase, histogram času odpovědí: vlevo bez repetitivních či v čase nahuštěných odpovědí u konce platnosti ankety, vpravo všechny odpovědi.}
\label{fig:troll_indication}
\end{figure}


%%%%%%%%%%%%%%%%%%%%%%%%%%
\section{Testy stability}

\subsection{Časový vývoj}
\htmladdnormallink{\textcolor{blue}{\underline{Animované GIFy}}}{http://jointlab.upol.cz/kvita/dotaznik/anim/} -- doporučujeme prohlédnout si v prohlížeči na pevném internetovém připojení, jde o stránku s cca 100 MB animovaných GIFů:) Odpovědi jsou stabilní v čase a i vůči tomu, kdy se odkaz na anketu objevil nejprve na UP reflexi a později na dŽurnálu, což indikuje i časová stabilita výsledků vůči tomu, odkud se respondenti o anketě dozvěděli. Velmi podobné výsledky jsou znát již od stadia 200 respondentů až k závěru.

\subsection{Srovnání výsledků po odstranění trollingu a po odstranění duplicit}
Srovnání výsledků po odstranění trollingu s případem po odstranění jen jasně duplicitních odpovědí viz Doplněk~\ref{sec:trolling}.

\subsection{Srovnání výsledků s a bez komentáře}
Srovnání výsledků s a bez komentáře viz Doplněk~\ref{sec:comments} po odstranění trollingu.

%%%%%%%%%%%%%%%%%%%%%%%%%%
\section{Osobní komentář}
Je smutné, k jak negativním emocem, trollingu až nenávisti jsme v komentářích místy došli.
Pokud má zřejmě jedna konkrétní osoba zájem, nabízím osobní setkání, nebo ať se prosím nad sebou alespoň zamyslí, zda trollení dotazníku je čestným způsobem, jak podpořit vznik VŠÚ. Např. v intervalu půldruhé minuty někdo evidentně cílil anketu změnit od 28.02.2020 10:12:18, a dále od 28.02.2020 18:01:20 apod.
%Zřejmě nevěděl(a), že graficky výsledky zpracujeme sami a že je velmi jednoduché trollení poznat dle komentářů, identičnosti odpovědí, jejich konverzačního charakteru či časové značky.
Pokud někdo anketu neuvítal, stačilo zaškrtnout příslušnou položku. Zde bychom rádi podotkli, že jméno jednoho z autorů jest Kvita a nikoli Kvíta, jak 3x jeden z respondentů uvedl s dalšími injektivami v komentářích:-) \\ S pozdravem, Jiří Dno Kocour Kuriozita:)

Současně z komentářů zaznívá mnoho hořkosti, deziluze, zklamání, rozpolcenosti, únavy, zášti, ale i podpory a naděje.

%%%%%%%%%%%%%%%%%%%%%%%%%%
\section{Výsledky dotazníku}
Výsledky ankety byly dle CVS souboru staženého z Google Forms analyzovány vlastním SW v jazyce {\tt Python3} a použitím knihoven {\tt pyplot}.
Barvy odpovědí jsou dány sestupným pořadím v jejich četnosti.
Bylo ověřeno, že celkové (zaujaté) výsledky dávají stejné grafické "koláčové grafy'' jako je vygenerovaný grafický výstup ankety na Google Forms.
Kód je \htmladdnormallink{\textcolor{blue}{\underline{k dispozici zde}}}{http://jointlab.upol.cz/kvita/dotaznik/code} (nebyl však testován mimo OS Linux).

\subsection{Komentář k výsledkům}

Výsledky po odstranění trollingu.

\noindent {\bf Demografie} -- dotazník vyplnilo od soboty 22.2.2020 10:00 do nedělního večera 1.3.2020 757 respondentů, z nichž 672 bylo shledáno unikátních, viz diskuze v Sekci~\ref{sec:Sel}.
\begin{itemize}
  \item 31\% respondentů tvoří ženy.
  \item 60\% respondentů je z PřF, po 10\% pak z LF a FF či RUP + ostatní složky fakulty, 10\% osoby mimo UP.
  \item 16\%, 8\% a 4\% tvoří PhD, Mgr a Bc studenti.
  \item 11\% jsou mladší a 37\% starší výzkumníci, 16\% THP/administrativní pracovníci.
  \item Přes 80\% respondentů anketu uvítala, a přes 90\% chce znát výsledky.
  \item komentář připojila $1/2$ respondentů.
\end{itemize}

\noindent {\bf Vybrané odpovědi na konkrétní otázky}
\begin{itemize}
  \item Dostatečně informovaných (UP informačními kanály) se cítí 60\% respondentů, 40\% málo či vůbec.
  \item 85\% respondentů  dění okolo VŠÚ považuje za celouniverzitní záležitost, jen 8\% za věc několika fakult, další za záležitost celé ČR či české vědy.
  \item 84\% je nespokojeno se způsobem řešení vzniku VŠÚ, spokojeno je jen 11\%.
  \item 45\% respondentů by váhalo projevit veřejně a neanonymně svůj názor, 48\% by neváhalo, zbytek dle tématu.
  \item 77\% respondentů se domnívá, že některé etické kauzy je potřeba vyřešit před vznikem VŠÚ, 16\% naopak.
  \item Současné etické kauzy nepokládá za dobře řešené 89\% respondentů.
  \item $1/2$ by uvítala univerzitního ombudsmana, $1/4$ nemá vyhraněný názor.
  \item 75\% bohužel považuje vedení UP za zaujaté, a to jak v etických kauzách, tak v řešení vzniku VŠÚ.
  \item Transparentnost u vedení UP shledává jen 17\% respondentů, 77\% naopak.
  \item 52\% (33\%) vidí UP jako místo s místem vážnými (určitými) problémy, jen 6\% bez problémů.
{\bf 
  \item Srovnání výsledků ohledně vzniku VŠÚ:
    \begin{itemize} 
      \item[$\circ$] Po odstranění trollingu:
      \\ Pro vznik VŠÚ je 19\%, dalších 29\% po dořešení otázek, proti je 40\%, \\ jasný názor nemá 12\%.
      \item[$\circ$] Výsledky na neduplicitní množině
      \\ Pro vznik VŠÚ je 25\%, dalších 27\% po dořešení otázek, proti je 38\%, \\ jasný názor nemá 11\%.
      \item[$\circ$] Trollovaná plná verze ankety
      \\ Pro vznik VŠÚ je 28\%, dalších 25\% po dořešení otázek, proti je 36\%, \\ jasný názor nemá 11\%.
    \end{itemize}
}
  \item 30\% má pocit, že se může pro vstup svobodně rozhodnout, 12\% že ne, 57\% neví/netýká se jich to.
{\bf 
  \item Srovnání výsledků ohledně vstupu do VŠÚ: Pokud by měli možnost, tak do VŠÚ
    \begin{itemize}
       \item[$\circ$] Po odstranění trollingu:
       \\  chce vstoupit 15\% (101 osob z 672), 10\% ne za současných podmínek, 31\% nechce, 39\% neví/netýká se jich to.
       \item[$\circ$] Výsledky na neduplicitní množině:
       \\  chce vstoupit 20\% (147 osob z 724), 10\% ne za současných podmínek, 29\% nechce, 37\% neví/netýká se jich to.
       \item[$\circ$] Trollovaná plná verze ankety:
         \\ chce vstoupit 24\% (179 osob z 757), 9\% ne za současných podmínek, 28\% nechce, 35\% neví/netýká se jich to.
    \end{itemize}
}
  \item Komentáře dále také odhalují strach respondentů se veřejně vyjádřit, a většinou i uvítání ankety, ale také kritické připomínky k volbě otázek, zatímco jiní zase otázky vítají.
\end{itemize}

\subsection{Hlavní výsledky}
Hlavní výsledky jsou uvedeny v Sekci~\ref{sec:results}.

\bigskip

\noindent Snad bude akademické obci a vedení UP anketa jako jistý barometr nálady na UP užitečná.

\bigskip

Se srdečným pozdravem

V Olomouci dne 4.3.2020

Martin Fafejta a Jiří Kvita

\subsection{Changelog}
Releases
\begin{itemize}
  \item 4.3.2020: First release.
  \item 5.3.2020: Rounding fix in absolute numbers in pie charts and removed 8 empty responses. No major impact on numbers.
\end{itemize}

\newpage
\section{Hlavní výsledky}
\label{sec:results}
% _____________________________________________________________________ %
{  \centerline{\includegraphics[width=0.99\textwidth]{ResultsNonTrolled/pdf_All/00.pdf}  } }


% _____________________________________________________________________ %
{  \centerline{\includegraphics[width=0.99\textwidth]{ResultsNonTrolled/pdf_All/01.pdf}  } }


% _____________________________________________________________________ %
{  \centerline{\includegraphics[width=0.99\textwidth]{ResultsNonTrolled/pdf_All/02.pdf}  } }


% _____________________________________________________________________ %
{  \centerline{\includegraphics[width=0.99\textwidth]{ResultsNonTrolled/pdf_All/03.pdf}  } }


% _____________________________________________________________________ %
{  \centerline{\includegraphics[width=0.99\textwidth]{ResultsNonTrolled/pdf_All/04.pdf}  } }


% _____________________________________________________________________ %
{  \centerline{\includegraphics[width=0.99\textwidth]{ResultsNonTrolled/pdf_All/05.pdf}  } }


% _____________________________________________________________________ %
{  \centerline{\includegraphics[width=0.99\textwidth]{ResultsNonTrolled/pdf_All/06.pdf}  } }


% _____________________________________________________________________ %
{  \centerline{\includegraphics[width=0.99\textwidth]{ResultsNonTrolled/pdf_All/07.pdf}  } }


% _____________________________________________________________________ %
{  \centerline{\includegraphics[width=0.99\textwidth]{ResultsNonTrolled/pdf_All/08.pdf}  } }


% _____________________________________________________________________ %
{  \centerline{\includegraphics[width=0.99\textwidth]{ResultsNonTrolled/pdf_All/09.pdf}  } }


% _____________________________________________________________________ %
{  \centerline{\includegraphics[width=0.99\textwidth]{ResultsNonTrolled/pdf_All/10.pdf}  } }


% _____________________________________________________________________ %
{  \centerline{\includegraphics[width=0.99\textwidth]{ResultsNonTrolled/pdf_All/11.pdf}  } }


% _____________________________________________________________________ %
{  \centerline{\includegraphics[width=0.99\textwidth]{ResultsNonTrolled/pdf_All/12.pdf}  } }


% _____________________________________________________________________ %
{  \centerline{\includegraphics[width=0.99\textwidth]{ResultsNonTrolled/pdf_All/13.pdf}  } }


% _____________________________________________________________________ %
{  \centerline{\includegraphics[width=0.99\textwidth]{ResultsNonTrolled/pdf_All/14.pdf}  } }


% _____________________________________________________________________ %
{  \centerline{\includegraphics[width=0.99\textwidth]{ResultsNonTrolled/pdf_All/15.pdf}  } }


% _____________________________________________________________________ %
{  \centerline{\includegraphics[width=0.99\textwidth]{ResultsNonTrolled/pdf_All/16.pdf}  } }


% _____________________________________________________________________ %
{  \centerline{\includegraphics[width=0.99\textwidth]{ResultsNonTrolled/pdf_All/17.pdf}  } }


% _____________________________________________________________________ %
{  \centerline{\includegraphics[width=0.99\textwidth]{ResultsNonTrolled/pdf_All/18.pdf}  } }


% _____________________________________________________________________ %
{  \centerline{\includegraphics[width=0.99\textwidth]{ResultsNonTrolled/pdf_All/19.pdf}  } }


% _____________________________________________________________________ %
{  \centerline{\includegraphics[width=0.99\textwidth]{ResultsNonTrolled/pdf_All/20.pdf}  } }


% _____________________________________________________________________ %
{  \centerline{\includegraphics[width=0.99\textwidth]{ResultsNonTrolled/pdf_All/21.pdf}  } }


% _____________________________________________________________________ %
{  \centerline{\includegraphics[width=0.99\textwidth]{ResultsNonTrolled/pdf_All/22.pdf}  } }


% _____________________________________________________________________ %
{  \centerline{\includegraphics[width=0.99\textwidth]{ResultsNonTrolled/pdf_All/23.pdf}  } }


% _____________________________________________________________________ %
{  \centerline{\includegraphics[width=0.99\textwidth]{ResultsNonTrolled/pdf_All/24.pdf}  } }


% _____________________________________________________________________ %
{  \centerline{\includegraphics[width=0.99\textwidth]{ResultsNonTrolled/pdf_All/25.pdf}  } }


% _____________________________________________________________________ %
{  \centerline{\includegraphics[width=0.99\textwidth]{ResultsNonTrolled/pdf_All/26.pdf}  } }


% _____________________________________________________________________ %
{  \centerline{\includegraphics[width=0.99\textwidth]{ResultsNonTrolled/pdf_All/27.pdf}  } }


% _____________________________________________________________________ %
{  \centerline{\includegraphics[width=0.99\textwidth]{ResultsNonTrolled/pdf_All/28.pdf}  } }


% _____________________________________________________________________ %
{  \centerline{\includegraphics[width=0.99\textwidth]{ResultsNonTrolled/pdf_All/29.pdf}  } }


% _____________________________________________________________________ %
{  \centerline{\includegraphics[width=0.99\textwidth]{ResultsNonTrolled/pdf_All/30.pdf}  } }


% _____________________________________________________________________ %
{  \centerline{\includegraphics[width=0.99\textwidth]{ResultsNonTrolled/pdf_All/31.pdf}  } }





\appendix

\newpage
%%%%%%%%%%%%%%%%%%%%%%%%%%
\section{Úvodní komentář ankety}
Dotazník o dění na UP

\bigskip
\noindent Dotazník ohledně současného dění na UP, komunikace, diskutovaného vzniku Vysokoškolského ústavu (VŠÚ), etických kauz, role ombudsmana apod.

\bigskip
Vážené kolegyně a kolegové,

\bigskip
rádi bychom získali zpětnou vazbu z co nejširšího spektra akademické obce a pracovníků UP na současné dění na UP. Budeme velmi vděčni, pokud vyplníte následující dotazník, který je anonymní a jeho souhrnné výsledky budou zveřejněny po jeho uzavření. Prosíme Vás o jeho vyplnění do půlnoci 1. 3. 2020.

\bigskip
Budeme rádi, pokud budete odkaz na dotazník šířit dále mezi své kolegyně a kolegy.

\bigskip
S díky za Váš čas a spolupráci,

\bigskip
Jiří Kvita, Martin Fafejta

\bigskip
PS: Jakýkolik netriviální krátký komentář na konci ankety je nyní povinný pro lepši ověření autentičnosti jejího vyplnění:) Nenechte se tím prosím odradit a napište alespoň krátký osobní postřeh. S díky, JK

\newpage
%%%%%%%%%%%%%%%%%%%%%%%%%%
\section{Srovnání výsledků po odstranění trollingu a po odstranění duplicit}
\label{sec:trolling}
% _____________________________________________________________________ %
\begin{tabular}{cc}
{  \includegraphics[width=0.49\textwidth]{ResultsNonTrolled/pdf_All/00.pdf} } & 
{  \includegraphics[width=0.49\textwidth]{ResultsNonDouble/pdf_All/00.pdf}} \\
 non-trolled & non-repetitive \\
\end{tabular}

% _____________________________________________________________________ %
\begin{tabular}{cc}
{  \includegraphics[width=0.49\textwidth]{ResultsNonTrolled/pdf_All/01.pdf} } & 
{  \includegraphics[width=0.49\textwidth]{ResultsNonDouble/pdf_All/01.pdf}} \\
 non-trolled & non-repetitive \\
\end{tabular}

% _____________________________________________________________________ %
\begin{tabular}{cc}
{  \includegraphics[width=0.49\textwidth]{ResultsNonTrolled/pdf_All/02.pdf} } & 
{  \includegraphics[width=0.49\textwidth]{ResultsNonDouble/pdf_All/02.pdf}} \\
 non-trolled & non-repetitive \\
\end{tabular}

% _____________________________________________________________________ %
\begin{tabular}{cc}
{  \includegraphics[width=0.49\textwidth]{ResultsNonTrolled/pdf_All/03.pdf} } & 
{  \includegraphics[width=0.49\textwidth]{ResultsNonDouble/pdf_All/03.pdf}} \\
 non-trolled & non-repetitive \\
\end{tabular}

% _____________________________________________________________________ %
\begin{tabular}{cc}
{  \includegraphics[width=0.49\textwidth]{ResultsNonTrolled/pdf_All/04.pdf} } & 
{  \includegraphics[width=0.49\textwidth]{ResultsNonDouble/pdf_All/04.pdf}} \\
 non-trolled & non-repetitive \\
\end{tabular}

% _____________________________________________________________________ %
\begin{tabular}{cc}
{  \includegraphics[width=0.49\textwidth]{ResultsNonTrolled/pdf_All/05.pdf} } & 
{  \includegraphics[width=0.49\textwidth]{ResultsNonDouble/pdf_All/05.pdf}} \\
 non-trolled & non-repetitive \\
\end{tabular}

% _____________________________________________________________________ %
\begin{tabular}{cc}
{  \includegraphics[width=0.49\textwidth]{ResultsNonTrolled/pdf_All/06.pdf} } & 
{  \includegraphics[width=0.49\textwidth]{ResultsNonDouble/pdf_All/06.pdf}} \\
 non-trolled & non-repetitive \\
\end{tabular}

% _____________________________________________________________________ %
\begin{tabular}{cc}
{  \includegraphics[width=0.49\textwidth]{ResultsNonTrolled/pdf_All/07.pdf} } & 
{  \includegraphics[width=0.49\textwidth]{ResultsNonDouble/pdf_All/07.pdf}} \\
 non-trolled & non-repetitive \\
\end{tabular}

% _____________________________________________________________________ %
\begin{tabular}{cc}
{  \includegraphics[width=0.49\textwidth]{ResultsNonTrolled/pdf_All/08.pdf} } & 
{  \includegraphics[width=0.49\textwidth]{ResultsNonDouble/pdf_All/08.pdf}} \\
 non-trolled & non-repetitive \\
\end{tabular}

% _____________________________________________________________________ %
\begin{tabular}{cc}
{  \includegraphics[width=0.49\textwidth]{ResultsNonTrolled/pdf_All/09.pdf} } & 
{  \includegraphics[width=0.49\textwidth]{ResultsNonDouble/pdf_All/09.pdf}} \\
 non-trolled & non-repetitive \\
\end{tabular}

% _____________________________________________________________________ %
\begin{tabular}{cc}
{  \includegraphics[width=0.49\textwidth]{ResultsNonTrolled/pdf_All/10.pdf} } & 
{  \includegraphics[width=0.49\textwidth]{ResultsNonDouble/pdf_All/10.pdf}} \\
 non-trolled & non-repetitive \\
\end{tabular}

% _____________________________________________________________________ %
\begin{tabular}{cc}
{  \includegraphics[width=0.49\textwidth]{ResultsNonTrolled/pdf_All/11.pdf} } & 
{  \includegraphics[width=0.49\textwidth]{ResultsNonDouble/pdf_All/11.pdf}} \\
 non-trolled & non-repetitive \\
\end{tabular}

% _____________________________________________________________________ %
\begin{tabular}{cc}
{  \includegraphics[width=0.49\textwidth]{ResultsNonTrolled/pdf_All/12.pdf} } & 
{  \includegraphics[width=0.49\textwidth]{ResultsNonDouble/pdf_All/12.pdf}} \\
 non-trolled & non-repetitive \\
\end{tabular}

% _____________________________________________________________________ %
\begin{tabular}{cc}
{  \includegraphics[width=0.49\textwidth]{ResultsNonTrolled/pdf_All/13.pdf} } & 
{  \includegraphics[width=0.49\textwidth]{ResultsNonDouble/pdf_All/13.pdf}} \\
 non-trolled & non-repetitive \\
\end{tabular}

% _____________________________________________________________________ %
\begin{tabular}{cc}
{  \includegraphics[width=0.49\textwidth]{ResultsNonTrolled/pdf_All/14.pdf} } & 
{  \includegraphics[width=0.49\textwidth]{ResultsNonDouble/pdf_All/14.pdf}} \\
 non-trolled & non-repetitive \\
\end{tabular}

% _____________________________________________________________________ %
\begin{tabular}{cc}
{  \includegraphics[width=0.49\textwidth]{ResultsNonTrolled/pdf_All/15.pdf} } & 
{  \includegraphics[width=0.49\textwidth]{ResultsNonDouble/pdf_All/15.pdf}} \\
 non-trolled & non-repetitive \\
\end{tabular}

% _____________________________________________________________________ %
\begin{tabular}{cc}
{  \includegraphics[width=0.49\textwidth]{ResultsNonTrolled/pdf_All/16.pdf} } & 
{  \includegraphics[width=0.49\textwidth]{ResultsNonDouble/pdf_All/16.pdf}} \\
 non-trolled & non-repetitive \\
\end{tabular}

% _____________________________________________________________________ %
\begin{tabular}{cc}
{  \includegraphics[width=0.49\textwidth]{ResultsNonTrolled/pdf_All/17.pdf} } & 
{  \includegraphics[width=0.49\textwidth]{ResultsNonDouble/pdf_All/17.pdf}} \\
 non-trolled & non-repetitive \\
\end{tabular}

% _____________________________________________________________________ %
\begin{tabular}{cc}
{  \includegraphics[width=0.49\textwidth]{ResultsNonTrolled/pdf_All/18.pdf} } & 
{  \includegraphics[width=0.49\textwidth]{ResultsNonDouble/pdf_All/18.pdf}} \\
 non-trolled & non-repetitive \\
\end{tabular}

% _____________________________________________________________________ %
\begin{tabular}{cc}
{  \includegraphics[width=0.49\textwidth]{ResultsNonTrolled/pdf_All/19.pdf} } & 
{  \includegraphics[width=0.49\textwidth]{ResultsNonDouble/pdf_All/19.pdf}} \\
 non-trolled & non-repetitive \\
\end{tabular}

% _____________________________________________________________________ %
\begin{tabular}{cc}
{  \includegraphics[width=0.49\textwidth]{ResultsNonTrolled/pdf_All/20.pdf} } & 
{  \includegraphics[width=0.49\textwidth]{ResultsNonDouble/pdf_All/20.pdf}} \\
 non-trolled & non-repetitive \\
\end{tabular}

% _____________________________________________________________________ %
\begin{tabular}{cc}
{  \includegraphics[width=0.49\textwidth]{ResultsNonTrolled/pdf_All/21.pdf} } & 
{  \includegraphics[width=0.49\textwidth]{ResultsNonDouble/pdf_All/21.pdf}} \\
 non-trolled & non-repetitive \\
\end{tabular}

% _____________________________________________________________________ %
\begin{tabular}{cc}
{  \includegraphics[width=0.49\textwidth]{ResultsNonTrolled/pdf_All/22.pdf} } & 
{  \includegraphics[width=0.49\textwidth]{ResultsNonDouble/pdf_All/22.pdf}} \\
 non-trolled & non-repetitive \\
\end{tabular}

% _____________________________________________________________________ %
\begin{tabular}{cc}
{  \includegraphics[width=0.49\textwidth]{ResultsNonTrolled/pdf_All/23.pdf} } & 
{  \includegraphics[width=0.49\textwidth]{ResultsNonDouble/pdf_All/23.pdf}} \\
 non-trolled & non-repetitive \\
\end{tabular}

% _____________________________________________________________________ %
\begin{tabular}{cc}
{  \includegraphics[width=0.49\textwidth]{ResultsNonTrolled/pdf_All/24.pdf} } & 
{  \includegraphics[width=0.49\textwidth]{ResultsNonDouble/pdf_All/24.pdf}} \\
 non-trolled & non-repetitive \\
\end{tabular}

% _____________________________________________________________________ %
\begin{tabular}{cc}
{  \includegraphics[width=0.49\textwidth]{ResultsNonTrolled/pdf_All/25.pdf} } & 
{  \includegraphics[width=0.49\textwidth]{ResultsNonDouble/pdf_All/25.pdf}} \\
 non-trolled & non-repetitive \\
\end{tabular}

% _____________________________________________________________________ %
\begin{tabular}{cc}
{  \includegraphics[width=0.49\textwidth]{ResultsNonTrolled/pdf_All/26.pdf} } & 
{  \includegraphics[width=0.49\textwidth]{ResultsNonDouble/pdf_All/26.pdf}} \\
 non-trolled & non-repetitive \\
\end{tabular}

% _____________________________________________________________________ %
\begin{tabular}{cc}
{  \includegraphics[width=0.49\textwidth]{ResultsNonTrolled/pdf_All/27.pdf} } & 
{  \includegraphics[width=0.49\textwidth]{ResultsNonDouble/pdf_All/27.pdf}} \\
 non-trolled & non-repetitive \\
\end{tabular}

% _____________________________________________________________________ %
\begin{tabular}{cc}
{  \includegraphics[width=0.49\textwidth]{ResultsNonTrolled/pdf_All/28.pdf} } & 
{  \includegraphics[width=0.49\textwidth]{ResultsNonDouble/pdf_All/28.pdf}} \\
 non-trolled & non-repetitive \\
\end{tabular}

% _____________________________________________________________________ %
\begin{tabular}{cc}
{  \includegraphics[width=0.49\textwidth]{ResultsNonTrolled/pdf_All/29.pdf} } & 
{  \includegraphics[width=0.49\textwidth]{ResultsNonDouble/pdf_All/29.pdf}} \\
 non-trolled & non-repetitive \\
\end{tabular}

% _____________________________________________________________________ %
\begin{tabular}{cc}
{  \includegraphics[width=0.49\textwidth]{ResultsNonTrolled/pdf_All/30.pdf} } & 
{  \includegraphics[width=0.49\textwidth]{ResultsNonDouble/pdf_All/30.pdf}} \\
 non-trolled & non-repetitive \\
\end{tabular}

% _____________________________________________________________________ %
\begin{tabular}{cc}
{  \includegraphics[width=0.49\textwidth]{ResultsNonTrolled/pdf_All/31.pdf} } & 
{  \includegraphics[width=0.49\textwidth]{ResultsNonDouble/pdf_All/31.pdf}} \\
 non-trolled & non-repetitive \\
\end{tabular}




\newpage
%%%%%%%%%%%%%%%%%%%%%%%%%%
\section{Srovnání výsledků s a bez komentářů}
\label{sec:comments}
\begin{tabular}{cc}% _____________________________________________________________________ %
{  \includegraphics[width=0.49\textwidth]{pdf_Comments/01_Jsem_pracovník.pdf} } & 
{  \includegraphics[width=0.49\textwidth]{pdf_noComments/01_Jsem_pracovník.pdf}} \\
 With comments & Without comments \\
\end{tabular}

\begin{tabular}{cc}% _____________________________________________________________________ %
{  \includegraphics[width=0.49\textwidth]{pdf_Comments/02_Jsem.pdf} } & 
{  \includegraphics[width=0.49\textwidth]{pdf_noComments/02_Jsem.pdf}} \\
 With comments & Without comments \\
\end{tabular}

\begin{tabular}{cc}% _____________________________________________________________________ %
{  \includegraphics[width=0.49\textwidth]{pdf_Comments/02_Jsem_z_fakulty_nebo_odjinud.pdf} } & 
{  \includegraphics[width=0.49\textwidth]{pdf_noComments/02_Jsem_z_fakulty_nebo_odjinud.pdf}} \\
 With comments & Without comments \\
\end{tabular}

\begin{tabular}{cc}% _____________________________________________________________________ %
{  \includegraphics[width=0.49\textwidth]{pdf_Comments/03_O_dění_na_UP_jsem_vnitřními_informačními_kanály_informován_a_.pdf} } & 
{  \includegraphics[width=0.49\textwidth]{pdf_noComments/03_O_dění_na_UP_jsem_vnitřními_informačními_kanály_informován_a_.pdf}} \\
 With comments & Without comments \\
\end{tabular}

\begin{tabular}{cc}% _____________________________________________________________________ %
{  \includegraphics[width=0.49\textwidth]{pdf_Comments/04_Dění_na_UP_ohledně_vzniku_Vysokoškolského_ústavu__dále_VŠÚ_.pdf} } & 
{  \includegraphics[width=0.49\textwidth]{pdf_noComments/04_Dění_na_UP_ohledně_vzniku_Vysokoškolského_ústavu__dále_VŠÚ_.pdf}} \\
 With comments & Without comments \\
\end{tabular}

\begin{tabular}{cc}% _____________________________________________________________________ %
{  \includegraphics[width=0.49\textwidth]{pdf_Comments/05_Informace_získávám.pdf} } & 
{  \includegraphics[width=0.49\textwidth]{pdf_noComments/05_Informace_získávám.pdf}} \\
 With comments & Without comments \\
\end{tabular}

\begin{tabular}{cc}% _____________________________________________________________________ %
{  \includegraphics[width=0.49\textwidth]{pdf_Comments/05_Současné_dění_okolo_VŠÚ_na_UP_považuji_za_záležitost.pdf} } & 
{  \includegraphics[width=0.49\textwidth]{pdf_noComments/05_Současné_dění_okolo_VŠÚ_na_UP_považuji_za_záležitost.pdf}} \\
 With comments & Without comments \\
\end{tabular}

\begin{tabular}{cc}% _____________________________________________________________________ %
{  \includegraphics[width=0.49\textwidth]{pdf_Comments/06_Se_způsobem__jakým_se_na_UP_řeší_vznik_VŠÚ__jsem.pdf} } & 
{  \includegraphics[width=0.49\textwidth]{pdf_noComments/06_Se_způsobem__jakým_se_na_UP_řeší_vznik_VŠÚ__jsem.pdf}} \\
 With comments & Without comments \\
\end{tabular}

\begin{tabular}{cc}% _____________________________________________________________________ %
{  \includegraphics[width=0.49\textwidth]{pdf_Comments/07_Na_dění_na_UP_ohledně_vzniku_VŠÚ.pdf} } & 
{  \includegraphics[width=0.49\textwidth]{pdf_noComments/07_Na_dění_na_UP_ohledně_vzniku_VŠÚ.pdf}} \\
 With comments & Without comments \\
\end{tabular}

\begin{tabular}{cc}% _____________________________________________________________________ %
{  \includegraphics[width=0.49\textwidth]{pdf_Comments/08_Na_dění_na_UP_ohledně_etických_kauz.pdf} } & 
{  \includegraphics[width=0.49\textwidth]{pdf_noComments/08_Na_dění_na_UP_ohledně_etických_kauz.pdf}} \\
 With comments & Without comments \\
\end{tabular}

\begin{tabular}{cc}% _____________________________________________________________________ %
{  \includegraphics[width=0.49\textwidth]{pdf_Comments/09_K_dění_na_UP_ohledně_vzniku_VŠÚ.pdf} } & 
{  \includegraphics[width=0.49\textwidth]{pdf_noComments/09_K_dění_na_UP_ohledně_vzniku_VŠÚ.pdf}} \\
 With comments & Without comments \\
\end{tabular}

\begin{tabular}{cc}% _____________________________________________________________________ %
{  \includegraphics[width=0.49\textwidth]{pdf_Comments/10_K_dění_na_UP_v_etických_kauzách.pdf} } & 
{  \includegraphics[width=0.49\textwidth]{pdf_noComments/10_K_dění_na_UP_v_etických_kauzách.pdf}} \\
 With comments & Without comments \\
\end{tabular}

\begin{tabular}{cc}% _____________________________________________________________________ %
{  \includegraphics[width=0.49\textwidth]{pdf_Comments/11_Svůj_názor_na_dění_na_UP_ohledně_vzniku_VŠÚ.pdf} } & 
{  \includegraphics[width=0.49\textwidth]{pdf_noComments/11_Svůj_názor_na_dění_na_UP_ohledně_vzniku_VŠÚ.pdf}} \\
 With comments & Without comments \\
\end{tabular}

\begin{tabular}{cc}% _____________________________________________________________________ %
{  \includegraphics[width=0.49\textwidth]{pdf_Comments/12_Svůj_názor_na_dění_na_UP_ohledně_etických_kauz.pdf} } & 
{  \includegraphics[width=0.49\textwidth]{pdf_noComments/12_Svůj_názor_na_dění_na_UP_ohledně_etických_kauz.pdf}} \\
 With comments & Without comments \\
\end{tabular}

\begin{tabular}{cc}% _____________________________________________________________________ %
{  \includegraphics[width=0.49\textwidth]{pdf_Comments/13_Domníváte_se__že_některé_etické_kauzy_je_potřeba_vyřešit_před_vznikem_VŠÚ_.pdf} } & 
{  \includegraphics[width=0.49\textwidth]{pdf_noComments/13_Domníváte_se__že_některé_etické_kauzy_je_potřeba_vyřešit_před_vznikem_VŠÚ_.pdf}} \\
 With comments & Without comments \\
\end{tabular}

\begin{tabular}{cc}% _____________________________________________________________________ %
{  \includegraphics[width=0.49\textwidth]{pdf_Comments/14_Současné_etické_kauzy_na_UP_I.pdf} } & 
{  \includegraphics[width=0.49\textwidth]{pdf_noComments/14_Současné_etické_kauzy_na_UP_I.pdf}} \\
 With comments & Without comments \\
\end{tabular}

\begin{tabular}{cc}% _____________________________________________________________________ %
{  \includegraphics[width=0.49\textwidth]{pdf_Comments/15_Současné_etické_kauzy_na_UP_II.pdf} } & 
{  \includegraphics[width=0.49\textwidth]{pdf_noComments/15_Současné_etické_kauzy_na_UP_II.pdf}} \\
 With comments & Without comments \\
\end{tabular}

\begin{tabular}{cc}% _____________________________________________________________________ %
{  \includegraphics[width=0.49\textwidth]{pdf_Comments/16_Dění_okolo_Etické_komise_UP.pdf} } & 
{  \includegraphics[width=0.49\textwidth]{pdf_noComments/16_Dění_okolo_Etické_komise_UP.pdf}} \\
 With comments & Without comments \\
\end{tabular}

\begin{tabular}{cc}% _____________________________________________________________________ %
{  \includegraphics[width=0.49\textwidth]{pdf_Comments/17_Etické_podněty_ohledně_možného_falšování_dat_považuji_za.pdf} } & 
{  \includegraphics[width=0.49\textwidth]{pdf_noComments/17_Etické_podněty_ohledně_možného_falšování_dat_považuji_za.pdf}} \\
 With comments & Without comments \\
\end{tabular}

\begin{tabular}{cc}% _____________________________________________________________________ %
{  \includegraphics[width=0.49\textwidth]{pdf_Comments/18_Etické_podněty_k_chování__komunikaci__práci_či_rozhodování_některých_zaměstnanců_či_vedoucích_pracovníků_považuji_za.pdf} } & 
{  \includegraphics[width=0.49\textwidth]{pdf_noComments/18_Etické_podněty_k_chování__komunikaci__práci_či_rozhodování_některých_zaměstnanců_či_vedoucích_pracovníků_považuji_za.pdf}} \\
 With comments & Without comments \\
\end{tabular}

\begin{tabular}{cc}% _____________________________________________________________________ %
{  \includegraphics[width=0.49\textwidth]{pdf_Comments/19_Roli_univerzitního_ombudsmananeboky.pdf} } & 
{  \includegraphics[width=0.49\textwidth]{pdf_noComments/19_Roli_univerzitního_ombudsmananeboky.pdf}} \\
 With comments & Without comments \\
\end{tabular}

\begin{tabular}{cc}% _____________________________________________________________________ %
{  \includegraphics[width=0.49\textwidth]{pdf_Comments/20_Postoje_a_postup_vedení_UP_v_řešení_vzniku_VŠÚ_považuji_za.pdf} } & 
{  \includegraphics[width=0.49\textwidth]{pdf_noComments/20_Postoje_a_postup_vedení_UP_v_řešení_vzniku_VŠÚ_považuji_za.pdf}} \\
 With comments & Without comments \\
\end{tabular}

\begin{tabular}{cc}% _____________________________________________________________________ %
{  \includegraphics[width=0.49\textwidth]{pdf_Comments/21_Postoje_a_postup_vedení_UP_v_řešení_etických_kauz_považuji_za.pdf} } & 
{  \includegraphics[width=0.49\textwidth]{pdf_noComments/21_Postoje_a_postup_vedení_UP_v_řešení_etických_kauz_považuji_za.pdf}} \\
 With comments & Without comments \\
\end{tabular}

\begin{tabular}{cc}% _____________________________________________________________________ %
{  \includegraphics[width=0.49\textwidth]{pdf_Comments/22_Způsob_řešení_problémů_na_UP_ze_strany_vedení_UP_považuji_za_.pdf} } & 
{  \includegraphics[width=0.49\textwidth]{pdf_noComments/22_Způsob_řešení_problémů_na_UP_ze_strany_vedení_UP_považuji_za_.pdf}} \\
 With comments & Without comments \\
\end{tabular}

\begin{tabular}{cc}% _____________________________________________________________________ %
{  \includegraphics[width=0.49\textwidth]{pdf_Comments/23_Univerzitu_Palackého_považuji_z_hlediska_mé_vědecké__pedagogické_či_jiné_práce.pdf} } & 
{  \includegraphics[width=0.49\textwidth]{pdf_noComments/23_Univerzitu_Palackého_považuji_z_hlediska_mé_vědecké__pedagogické_či_jiné_práce.pdf}} \\
 With comments & Without comments \\
\end{tabular}

\begin{tabular}{cc}% _____________________________________________________________________ %
{  \includegraphics[width=0.49\textwidth]{pdf_Comments/24_Ad_případný_vznik_VŠÚ_na_UP.pdf} } & 
{  \includegraphics[width=0.49\textwidth]{pdf_noComments/24_Ad_případný_vznik_VŠÚ_na_UP.pdf}} \\
 With comments & Without comments \\
\end{tabular}

\begin{tabular}{cc}% _____________________________________________________________________ %
{  \includegraphics[width=0.49\textwidth]{pdf_Comments/25_Mám_možnost_se_svobodně_rozhodnout__zda_ve_VŠÚ_chci_pracovat_nebo_ne.pdf} } & 
{  \includegraphics[width=0.49\textwidth]{pdf_noComments/25_Mám_možnost_se_svobodně_rozhodnout__zda_ve_VŠÚ_chci_pracovat_nebo_ne.pdf}} \\
 With comments & Without comments \\
\end{tabular}

\begin{tabular}{cc}% _____________________________________________________________________ %
{  \includegraphics[width=0.49\textwidth]{pdf_Comments/26_Pokud_bych_měl_možnost_se_rozhodnout__pak_bych_do_VŠÚ_.pdf} } & 
{  \includegraphics[width=0.49\textwidth]{pdf_noComments/26_Pokud_bych_měl_možnost_se_rozhodnout__pak_bych_do_VŠÚ_.pdf}} \\
 With comments & Without comments \\
\end{tabular}

\begin{tabular}{cc}% _____________________________________________________________________ %
{  \includegraphics[width=0.49\textwidth]{pdf_Comments/27_Anketu_jsem_uvítal_a_.pdf} } & 
{  \includegraphics[width=0.49\textwidth]{pdf_noComments/27_Anketu_jsem_uvítal_a_.pdf}} \\
 With comments & Without comments \\
\end{tabular}

\begin{tabular}{cc}% _____________________________________________________________________ %
{  \includegraphics[width=0.49\textwidth]{pdf_Comments/27_O_anketě_jsem_se_dozvěděl_a__.pdf} } & 
{  \includegraphics[width=0.49\textwidth]{pdf_noComments/27_O_anketě_jsem_se_dozvěděl_a__.pdf}} \\
 With comments & Without comments \\
\end{tabular}

\begin{tabular}{cc}% _____________________________________________________________________ %
{  \includegraphics[width=0.49\textwidth]{pdf_Comments/28_Rád_budu_seznámen_a__s_výsledky__budou_zveřejněny_na_UP_reflexi_.pdf} } & 
{  \includegraphics[width=0.49\textwidth]{pdf_noComments/28_Rád_budu_seznámen_a__s_výsledky__budou_zveřejněny_na_UP_reflexi_.pdf}} \\
 With comments & Without comments \\
\end{tabular}



\end{document}

%%%%%%%%%%%%%%%%%%%%%%%%%%
%%%%%%%%%%%%%%%%%%%%%%%%%%
%%%%%%%%%%%%%%%%%%%%%%%%%%
